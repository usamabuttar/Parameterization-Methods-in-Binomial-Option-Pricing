\documentclass[11pt]{article}

    \usepackage[breakable]{tcolorbox}
    \usepackage{parskip} % Stop auto-indenting (to mimic markdown behaviour)
    

    % Basic figure setup, for now with no caption control since it's done
    % automatically by Pandoc (which extracts ![](path) syntax from Markdown).
    \usepackage{graphicx}
    % Keep aspect ratio if custom image width or height is specified
    \setkeys{Gin}{keepaspectratio}
    % Maintain compatibility with old templates. Remove in nbconvert 6.0
    \let\Oldincludegraphics\includegraphics
    % Ensure that by default, figures have no caption (until we provide a
    % proper Figure object with a Caption API and a way to capture that
    % in the conversion process - todo).
    \usepackage{caption}
    \DeclareCaptionFormat{nocaption}{}
    \captionsetup{format=nocaption,aboveskip=0pt,belowskip=0pt}

    \usepackage{float}
    \floatplacement{figure}{H} % forces figures to be placed at the correct location
    \usepackage{xcolor} % Allow colors to be defined
    \usepackage{enumerate} % Needed for markdown enumerations to work
    \usepackage{geometry} % Used to adjust the document margins
    \usepackage{amsmath} % Equations
    \usepackage{amssymb} % Equations
    \usepackage{textcomp} % defines textquotesingle
    % Hack from http://tex.stackexchange.com/a/47451/13684:
    \AtBeginDocument{%
        \def\PYZsq{\textquotesingle}% Upright quotes in Pygmentized code
    }
    \usepackage{upquote} % Upright quotes for verbatim code
    \usepackage{eurosym} % defines \euro

    \usepackage{iftex}
    \ifPDFTeX
        \usepackage[T1]{fontenc}
        \IfFileExists{alphabeta.sty}{
              \usepackage{alphabeta}
          }{
              \usepackage[mathletters]{ucs}
              \usepackage[utf8x]{inputenc}
          }
    \else
        \usepackage{fontspec}
        \usepackage{unicode-math}
    \fi

    \usepackage{fancyvrb} % verbatim replacement that allows latex
    \usepackage{grffile} % extends the file name processing of package graphics
                         % to support a larger range
    \makeatletter % fix for old versions of grffile with XeLaTeX
    \@ifpackagelater{grffile}{2019/11/01}
    {
      % Do nothing on new versions
    }
    {
      \def\Gread@@xetex#1{%
        \IfFileExists{"\Gin@base".bb}%
        {\Gread@eps{\Gin@base.bb}}%
        {\Gread@@xetex@aux#1}%
      }
    }
    \makeatother
    \usepackage[Export]{adjustbox} % Used to constrain images to a maximum size
    \adjustboxset{max size={0.9\linewidth}{0.9\paperheight}}

    % The hyperref package gives us a pdf with properly built
    % internal navigation ('pdf bookmarks' for the table of contents,
    % internal cross-reference links, web links for URLs, etc.)
    \usepackage{hyperref}
    % The default LaTeX title has an obnoxious amount of whitespace. By default,
    % titling removes some of it. It also provides customization options.
    \usepackage{titling}
    \usepackage{longtable} % longtable support required by pandoc >1.10
    \usepackage{booktabs}  % table support for pandoc > 1.12.2
    \usepackage{array}     % table support for pandoc >= 2.11.3
    \usepackage{calc}      % table minipage width calculation for pandoc >= 2.11.1
    \usepackage[inline]{enumitem} % IRkernel/repr support (it uses the enumerate* environment)
    \usepackage[normalem]{ulem} % ulem is needed to support strikethroughs (\sout)
                                % normalem makes italics be italics, not underlines
    \usepackage{soul}      % strikethrough (\st) support for pandoc >= 3.0.0
    \usepackage{mathrsfs}
    

    
    % Colors for the hyperref package
    \definecolor{urlcolor}{rgb}{0,.145,.698}
    \definecolor{linkcolor}{rgb}{.71,0.21,0.01}
    \definecolor{citecolor}{rgb}{.12,.54,.11}

    % ANSI colors
    \definecolor{ansi-black}{HTML}{3E424D}
    \definecolor{ansi-black-intense}{HTML}{282C36}
    \definecolor{ansi-red}{HTML}{E75C58}
    \definecolor{ansi-red-intense}{HTML}{B22B31}
    \definecolor{ansi-green}{HTML}{00A250}
    \definecolor{ansi-green-intense}{HTML}{007427}
    \definecolor{ansi-yellow}{HTML}{DDB62B}
    \definecolor{ansi-yellow-intense}{HTML}{B27D12}
    \definecolor{ansi-blue}{HTML}{208FFB}
    \definecolor{ansi-blue-intense}{HTML}{0065CA}
    \definecolor{ansi-magenta}{HTML}{D160C4}
    \definecolor{ansi-magenta-intense}{HTML}{A03196}
    \definecolor{ansi-cyan}{HTML}{60C6C8}
    \definecolor{ansi-cyan-intense}{HTML}{258F8F}
    \definecolor{ansi-white}{HTML}{C5C1B4}
    \definecolor{ansi-white-intense}{HTML}{A1A6B2}
    \definecolor{ansi-default-inverse-fg}{HTML}{FFFFFF}
    \definecolor{ansi-default-inverse-bg}{HTML}{000000}

    % common color for the border for error outputs.
    \definecolor{outerrorbackground}{HTML}{FFDFDF}

    % commands and environments needed by pandoc snippets
    % extracted from the output of `pandoc -s`
    \providecommand{\tightlist}{%
      \setlength{\itemsep}{0pt}\setlength{\parskip}{0pt}}
    \DefineVerbatimEnvironment{Highlighting}{Verbatim}{commandchars=\\\{\}}
    % Add ',fontsize=\small' for more characters per line
    \newenvironment{Shaded}{}{}
    \newcommand{\KeywordTok}[1]{\textcolor[rgb]{0.00,0.44,0.13}{\textbf{{#1}}}}
    \newcommand{\DataTypeTok}[1]{\textcolor[rgb]{0.56,0.13,0.00}{{#1}}}
    \newcommand{\DecValTok}[1]{\textcolor[rgb]{0.25,0.63,0.44}{{#1}}}
    \newcommand{\BaseNTok}[1]{\textcolor[rgb]{0.25,0.63,0.44}{{#1}}}
    \newcommand{\FloatTok}[1]{\textcolor[rgb]{0.25,0.63,0.44}{{#1}}}
    \newcommand{\CharTok}[1]{\textcolor[rgb]{0.25,0.44,0.63}{{#1}}}
    \newcommand{\StringTok}[1]{\textcolor[rgb]{0.25,0.44,0.63}{{#1}}}
    \newcommand{\CommentTok}[1]{\textcolor[rgb]{0.38,0.63,0.69}{\textit{{#1}}}}
    \newcommand{\OtherTok}[1]{\textcolor[rgb]{0.00,0.44,0.13}{{#1}}}
    \newcommand{\AlertTok}[1]{\textcolor[rgb]{1.00,0.00,0.00}{\textbf{{#1}}}}
    \newcommand{\FunctionTok}[1]{\textcolor[rgb]{0.02,0.16,0.49}{{#1}}}
    \newcommand{\RegionMarkerTok}[1]{{#1}}
    \newcommand{\ErrorTok}[1]{\textcolor[rgb]{1.00,0.00,0.00}{\textbf{{#1}}}}
    \newcommand{\NormalTok}[1]{{#1}}

    % Additional commands for more recent versions of Pandoc
    \newcommand{\ConstantTok}[1]{\textcolor[rgb]{0.53,0.00,0.00}{{#1}}}
    \newcommand{\SpecialCharTok}[1]{\textcolor[rgb]{0.25,0.44,0.63}{{#1}}}
    \newcommand{\VerbatimStringTok}[1]{\textcolor[rgb]{0.25,0.44,0.63}{{#1}}}
    \newcommand{\SpecialStringTok}[1]{\textcolor[rgb]{0.73,0.40,0.53}{{#1}}}
    \newcommand{\ImportTok}[1]{{#1}}
    \newcommand{\DocumentationTok}[1]{\textcolor[rgb]{0.73,0.13,0.13}{\textit{{#1}}}}
    \newcommand{\AnnotationTok}[1]{\textcolor[rgb]{0.38,0.63,0.69}{\textbf{\textit{{#1}}}}}
    \newcommand{\CommentVarTok}[1]{\textcolor[rgb]{0.38,0.63,0.69}{\textbf{\textit{{#1}}}}}
    \newcommand{\VariableTok}[1]{\textcolor[rgb]{0.10,0.09,0.49}{{#1}}}
    \newcommand{\ControlFlowTok}[1]{\textcolor[rgb]{0.00,0.44,0.13}{\textbf{{#1}}}}
    \newcommand{\OperatorTok}[1]{\textcolor[rgb]{0.40,0.40,0.40}{{#1}}}
    \newcommand{\BuiltInTok}[1]{{#1}}
    \newcommand{\ExtensionTok}[1]{{#1}}
    \newcommand{\PreprocessorTok}[1]{\textcolor[rgb]{0.74,0.48,0.00}{{#1}}}
    \newcommand{\AttributeTok}[1]{\textcolor[rgb]{0.49,0.56,0.16}{{#1}}}
    \newcommand{\InformationTok}[1]{\textcolor[rgb]{0.38,0.63,0.69}{\textbf{\textit{{#1}}}}}
    \newcommand{\WarningTok}[1]{\textcolor[rgb]{0.38,0.63,0.69}{\textbf{\textit{{#1}}}}}


    % Define a nice break command that doesn't care if a line doesn't already
    % exist.
    \def\br{\hspace*{\fill} \\* }
    % Math Jax compatibility definitions
    \def\gt{>}
    \def\lt{<}
    \let\Oldtex\TeX
    \let\Oldlatex\LaTeX
    \renewcommand{\TeX}{\textrm{\Oldtex}}
    \renewcommand{\LaTeX}{\textrm{\Oldlatex}}
    % Document parameters
    % Document title
    \title{Parameterization Methods in Binomial Option Pricing}
    \author{Usama Buttar}
    
    
    
    
    
    
    
% Pygments definitions
\makeatletter
\def\PY@reset{\let\PY@it=\relax \let\PY@bf=\relax%
    \let\PY@ul=\relax \let\PY@tc=\relax%
    \let\PY@bc=\relax \let\PY@ff=\relax}
\def\PY@tok#1{\csname PY@tok@#1\endcsname}
\def\PY@toks#1+{\ifx\relax#1\empty\else%
    \PY@tok{#1}\expandafter\PY@toks\fi}
\def\PY@do#1{\PY@bc{\PY@tc{\PY@ul{%
    \PY@it{\PY@bf{\PY@ff{#1}}}}}}}
\def\PY#1#2{\PY@reset\PY@toks#1+\relax+\PY@do{#2}}

\@namedef{PY@tok@w}{\def\PY@tc##1{\textcolor[rgb]{0.73,0.73,0.73}{##1}}}
\@namedef{PY@tok@c}{\let\PY@it=\textit\def\PY@tc##1{\textcolor[rgb]{0.24,0.48,0.48}{##1}}}
\@namedef{PY@tok@cp}{\def\PY@tc##1{\textcolor[rgb]{0.61,0.40,0.00}{##1}}}
\@namedef{PY@tok@k}{\let\PY@bf=\textbf\def\PY@tc##1{\textcolor[rgb]{0.00,0.50,0.00}{##1}}}
\@namedef{PY@tok@kp}{\def\PY@tc##1{\textcolor[rgb]{0.00,0.50,0.00}{##1}}}
\@namedef{PY@tok@kt}{\def\PY@tc##1{\textcolor[rgb]{0.69,0.00,0.25}{##1}}}
\@namedef{PY@tok@o}{\def\PY@tc##1{\textcolor[rgb]{0.40,0.40,0.40}{##1}}}
\@namedef{PY@tok@ow}{\let\PY@bf=\textbf\def\PY@tc##1{\textcolor[rgb]{0.67,0.13,1.00}{##1}}}
\@namedef{PY@tok@nb}{\def\PY@tc##1{\textcolor[rgb]{0.00,0.50,0.00}{##1}}}
\@namedef{PY@tok@nf}{\def\PY@tc##1{\textcolor[rgb]{0.00,0.00,1.00}{##1}}}
\@namedef{PY@tok@nc}{\let\PY@bf=\textbf\def\PY@tc##1{\textcolor[rgb]{0.00,0.00,1.00}{##1}}}
\@namedef{PY@tok@nn}{\let\PY@bf=\textbf\def\PY@tc##1{\textcolor[rgb]{0.00,0.00,1.00}{##1}}}
\@namedef{PY@tok@ne}{\let\PY@bf=\textbf\def\PY@tc##1{\textcolor[rgb]{0.80,0.25,0.22}{##1}}}
\@namedef{PY@tok@nv}{\def\PY@tc##1{\textcolor[rgb]{0.10,0.09,0.49}{##1}}}
\@namedef{PY@tok@no}{\def\PY@tc##1{\textcolor[rgb]{0.53,0.00,0.00}{##1}}}
\@namedef{PY@tok@nl}{\def\PY@tc##1{\textcolor[rgb]{0.46,0.46,0.00}{##1}}}
\@namedef{PY@tok@ni}{\let\PY@bf=\textbf\def\PY@tc##1{\textcolor[rgb]{0.44,0.44,0.44}{##1}}}
\@namedef{PY@tok@na}{\def\PY@tc##1{\textcolor[rgb]{0.41,0.47,0.13}{##1}}}
\@namedef{PY@tok@nt}{\let\PY@bf=\textbf\def\PY@tc##1{\textcolor[rgb]{0.00,0.50,0.00}{##1}}}
\@namedef{PY@tok@nd}{\def\PY@tc##1{\textcolor[rgb]{0.67,0.13,1.00}{##1}}}
\@namedef{PY@tok@s}{\def\PY@tc##1{\textcolor[rgb]{0.73,0.13,0.13}{##1}}}
\@namedef{PY@tok@sd}{\let\PY@it=\textit\def\PY@tc##1{\textcolor[rgb]{0.73,0.13,0.13}{##1}}}
\@namedef{PY@tok@si}{\let\PY@bf=\textbf\def\PY@tc##1{\textcolor[rgb]{0.64,0.35,0.47}{##1}}}
\@namedef{PY@tok@se}{\let\PY@bf=\textbf\def\PY@tc##1{\textcolor[rgb]{0.67,0.36,0.12}{##1}}}
\@namedef{PY@tok@sr}{\def\PY@tc##1{\textcolor[rgb]{0.64,0.35,0.47}{##1}}}
\@namedef{PY@tok@ss}{\def\PY@tc##1{\textcolor[rgb]{0.10,0.09,0.49}{##1}}}
\@namedef{PY@tok@sx}{\def\PY@tc##1{\textcolor[rgb]{0.00,0.50,0.00}{##1}}}
\@namedef{PY@tok@m}{\def\PY@tc##1{\textcolor[rgb]{0.40,0.40,0.40}{##1}}}
\@namedef{PY@tok@gh}{\let\PY@bf=\textbf\def\PY@tc##1{\textcolor[rgb]{0.00,0.00,0.50}{##1}}}
\@namedef{PY@tok@gu}{\let\PY@bf=\textbf\def\PY@tc##1{\textcolor[rgb]{0.50,0.00,0.50}{##1}}}
\@namedef{PY@tok@gd}{\def\PY@tc##1{\textcolor[rgb]{0.63,0.00,0.00}{##1}}}
\@namedef{PY@tok@gi}{\def\PY@tc##1{\textcolor[rgb]{0.00,0.52,0.00}{##1}}}
\@namedef{PY@tok@gr}{\def\PY@tc##1{\textcolor[rgb]{0.89,0.00,0.00}{##1}}}
\@namedef{PY@tok@ge}{\let\PY@it=\textit}
\@namedef{PY@tok@gs}{\let\PY@bf=\textbf}
\@namedef{PY@tok@gp}{\let\PY@bf=\textbf\def\PY@tc##1{\textcolor[rgb]{0.00,0.00,0.50}{##1}}}
\@namedef{PY@tok@go}{\def\PY@tc##1{\textcolor[rgb]{0.44,0.44,0.44}{##1}}}
\@namedef{PY@tok@gt}{\def\PY@tc##1{\textcolor[rgb]{0.00,0.27,0.87}{##1}}}
\@namedef{PY@tok@err}{\def\PY@bc##1{{\setlength{\fboxsep}{\string -\fboxrule}\fcolorbox[rgb]{1.00,0.00,0.00}{1,1,1}{\strut ##1}}}}
\@namedef{PY@tok@kc}{\let\PY@bf=\textbf\def\PY@tc##1{\textcolor[rgb]{0.00,0.50,0.00}{##1}}}
\@namedef{PY@tok@kd}{\let\PY@bf=\textbf\def\PY@tc##1{\textcolor[rgb]{0.00,0.50,0.00}{##1}}}
\@namedef{PY@tok@kn}{\let\PY@bf=\textbf\def\PY@tc##1{\textcolor[rgb]{0.00,0.50,0.00}{##1}}}
\@namedef{PY@tok@kr}{\let\PY@bf=\textbf\def\PY@tc##1{\textcolor[rgb]{0.00,0.50,0.00}{##1}}}
\@namedef{PY@tok@bp}{\def\PY@tc##1{\textcolor[rgb]{0.00,0.50,0.00}{##1}}}
\@namedef{PY@tok@fm}{\def\PY@tc##1{\textcolor[rgb]{0.00,0.00,1.00}{##1}}}
\@namedef{PY@tok@vc}{\def\PY@tc##1{\textcolor[rgb]{0.10,0.09,0.49}{##1}}}
\@namedef{PY@tok@vg}{\def\PY@tc##1{\textcolor[rgb]{0.10,0.09,0.49}{##1}}}
\@namedef{PY@tok@vi}{\def\PY@tc##1{\textcolor[rgb]{0.10,0.09,0.49}{##1}}}
\@namedef{PY@tok@vm}{\def\PY@tc##1{\textcolor[rgb]{0.10,0.09,0.49}{##1}}}
\@namedef{PY@tok@sa}{\def\PY@tc##1{\textcolor[rgb]{0.73,0.13,0.13}{##1}}}
\@namedef{PY@tok@sb}{\def\PY@tc##1{\textcolor[rgb]{0.73,0.13,0.13}{##1}}}
\@namedef{PY@tok@sc}{\def\PY@tc##1{\textcolor[rgb]{0.73,0.13,0.13}{##1}}}
\@namedef{PY@tok@dl}{\def\PY@tc##1{\textcolor[rgb]{0.73,0.13,0.13}{##1}}}
\@namedef{PY@tok@s2}{\def\PY@tc##1{\textcolor[rgb]{0.73,0.13,0.13}{##1}}}
\@namedef{PY@tok@sh}{\def\PY@tc##1{\textcolor[rgb]{0.73,0.13,0.13}{##1}}}
\@namedef{PY@tok@s1}{\def\PY@tc##1{\textcolor[rgb]{0.73,0.13,0.13}{##1}}}
\@namedef{PY@tok@mb}{\def\PY@tc##1{\textcolor[rgb]{0.40,0.40,0.40}{##1}}}
\@namedef{PY@tok@mf}{\def\PY@tc##1{\textcolor[rgb]{0.40,0.40,0.40}{##1}}}
\@namedef{PY@tok@mh}{\def\PY@tc##1{\textcolor[rgb]{0.40,0.40,0.40}{##1}}}
\@namedef{PY@tok@mi}{\def\PY@tc##1{\textcolor[rgb]{0.40,0.40,0.40}{##1}}}
\@namedef{PY@tok@il}{\def\PY@tc##1{\textcolor[rgb]{0.40,0.40,0.40}{##1}}}
\@namedef{PY@tok@mo}{\def\PY@tc##1{\textcolor[rgb]{0.40,0.40,0.40}{##1}}}
\@namedef{PY@tok@ch}{\let\PY@it=\textit\def\PY@tc##1{\textcolor[rgb]{0.24,0.48,0.48}{##1}}}
\@namedef{PY@tok@cm}{\let\PY@it=\textit\def\PY@tc##1{\textcolor[rgb]{0.24,0.48,0.48}{##1}}}
\@namedef{PY@tok@cpf}{\let\PY@it=\textit\def\PY@tc##1{\textcolor[rgb]{0.24,0.48,0.48}{##1}}}
\@namedef{PY@tok@c1}{\let\PY@it=\textit\def\PY@tc##1{\textcolor[rgb]{0.24,0.48,0.48}{##1}}}
\@namedef{PY@tok@cs}{\let\PY@it=\textit\def\PY@tc##1{\textcolor[rgb]{0.24,0.48,0.48}{##1}}}

\def\PYZbs{\char`\\}
\def\PYZus{\char`\_}
\def\PYZob{\char`\{}
\def\PYZcb{\char`\}}
\def\PYZca{\char`\^}
\def\PYZam{\char`\&}
\def\PYZlt{\char`\<}
\def\PYZgt{\char`\>}
\def\PYZsh{\char`\#}
\def\PYZpc{\char`\%}
\def\PYZdl{\char`\$}
\def\PYZhy{\char`\-}
\def\PYZsq{\char`\'}
\def\PYZdq{\char`\"}
\def\PYZti{\char`\~}
% for compatibility with earlier versions
\def\PYZat{@}
\def\PYZlb{[}
\def\PYZrb{]}
\makeatother


    % For linebreaks inside Verbatim environment from package fancyvrb.
    \makeatletter
        \newbox\Wrappedcontinuationbox
        \newbox\Wrappedvisiblespacebox
        \newcommand*\Wrappedvisiblespace {\textcolor{red}{\textvisiblespace}}
        \newcommand*\Wrappedcontinuationsymbol {\textcolor{red}{\llap{\tiny$\m@th\hookrightarrow$}}}
        \newcommand*\Wrappedcontinuationindent {3ex }
        \newcommand*\Wrappedafterbreak {\kern\Wrappedcontinuationindent\copy\Wrappedcontinuationbox}
        % Take advantage of the already applied Pygments mark-up to insert
        % potential linebreaks for TeX processing.
        %        {, <, #, %, $, ' and ": go to next line.
        %        _, }, ^, &, >, - and ~: stay at end of broken line.
        % Use of \textquotesingle for straight quote.
        \newcommand*\Wrappedbreaksatspecials {%
            \def\PYGZus{\discretionary{\char`\_}{\Wrappedafterbreak}{\char`\_}}%
            \def\PYGZob{\discretionary{}{\Wrappedafterbreak\char`\{}{\char`\{}}%
            \def\PYGZcb{\discretionary{\char`\}}{\Wrappedafterbreak}{\char`\}}}%
            \def\PYGZca{\discretionary{\char`\^}{\Wrappedafterbreak}{\char`\^}}%
            \def\PYGZam{\discretionary{\char`\&}{\Wrappedafterbreak}{\char`\&}}%
            \def\PYGZlt{\discretionary{}{\Wrappedafterbreak\char`\<}{\char`\<}}%
            \def\PYGZgt{\discretionary{\char`\>}{\Wrappedafterbreak}{\char`\>}}%
            \def\PYGZsh{\discretionary{}{\Wrappedafterbreak\char`\#}{\char`\#}}%
            \def\PYGZpc{\discretionary{}{\Wrappedafterbreak\char`\%}{\char`\%}}%
            \def\PYGZdl{\discretionary{}{\Wrappedafterbreak\char`\$}{\char`\$}}%
            \def\PYGZhy{\discretionary{\char`\-}{\Wrappedafterbreak}{\char`\-}}%
            \def\PYGZsq{\discretionary{}{\Wrappedafterbreak\textquotesingle}{\textquotesingle}}%
            \def\PYGZdq{\discretionary{}{\Wrappedafterbreak\char`\"}{\char`\"}}%
            \def\PYGZti{\discretionary{\char`\~}{\Wrappedafterbreak}{\char`\~}}%
        }
        % Some characters . , ; ? ! / are not pygmentized.
        % This macro makes them "active" and they will insert potential linebreaks
        \newcommand*\Wrappedbreaksatpunct {%
            \lccode`\~`\.\lowercase{\def~}{\discretionary{\hbox{\char`\.}}{\Wrappedafterbreak}{\hbox{\char`\.}}}%
            \lccode`\~`\,\lowercase{\def~}{\discretionary{\hbox{\char`\,}}{\Wrappedafterbreak}{\hbox{\char`\,}}}%
            \lccode`\~`\;\lowercase{\def~}{\discretionary{\hbox{\char`\;}}{\Wrappedafterbreak}{\hbox{\char`\;}}}%
            \lccode`\~`\:\lowercase{\def~}{\discretionary{\hbox{\char`\:}}{\Wrappedafterbreak}{\hbox{\char`\:}}}%
            \lccode`\~`\?\lowercase{\def~}{\discretionary{\hbox{\char`\?}}{\Wrappedafterbreak}{\hbox{\char`\?}}}%
            \lccode`\~`\!\lowercase{\def~}{\discretionary{\hbox{\char`\!}}{\Wrappedafterbreak}{\hbox{\char`\!}}}%
            \lccode`\~`\/\lowercase{\def~}{\discretionary{\hbox{\char`\/}}{\Wrappedafterbreak}{\hbox{\char`\/}}}%
            \catcode`\.\active
            \catcode`\,\active
            \catcode`\;\active
            \catcode`\:\active
            \catcode`\?\active
            \catcode`\!\active
            \catcode`\/\active
            \lccode`\~`\~
        }
    \makeatother

    \let\OriginalVerbatim=\Verbatim
    \makeatletter
    \renewcommand{\Verbatim}[1][1]{%
        %\parskip\z@skip
        \sbox\Wrappedcontinuationbox {\Wrappedcontinuationsymbol}%
        \sbox\Wrappedvisiblespacebox {\FV@SetupFont\Wrappedvisiblespace}%
        \def\FancyVerbFormatLine ##1{\hsize\linewidth
            \vtop{\raggedright\hyphenpenalty\z@\exhyphenpenalty\z@
                \doublehyphendemerits\z@\finalhyphendemerits\z@
                \strut ##1\strut}%
        }%
        % If the linebreak is at a space, the latter will be displayed as visible
        % space at end of first line, and a continuation symbol starts next line.
        % Stretch/shrink are however usually zero for typewriter font.
        \def\FV@Space {%
            \nobreak\hskip\z@ plus\fontdimen3\font minus\fontdimen4\font
            \discretionary{\copy\Wrappedvisiblespacebox}{\Wrappedafterbreak}
            {\kern\fontdimen2\font}%
        }%

        % Allow breaks at special characters using \PYG... macros.
        \Wrappedbreaksatspecials
        % Breaks at punctuation characters . , ; ? ! and / need catcode=\active
        \OriginalVerbatim[#1,codes*=\Wrappedbreaksatpunct]%
    }
    \makeatother

    % Exact colors from NB
    \definecolor{incolor}{HTML}{303F9F}
    \definecolor{outcolor}{HTML}{D84315}
    \definecolor{cellborder}{HTML}{CFCFCF}
    \definecolor{cellbackground}{HTML}{F7F7F7}

    % prompt
    \makeatletter
    \newcommand{\boxspacing}{\kern\kvtcb@left@rule\kern\kvtcb@boxsep}
    \makeatother
    \newcommand{\prompt}[4]{
        {\ttfamily\llap{{\color{#2}[#3]:\hspace{3pt}#4}}\vspace{-\baselineskip}}
    }
    

    
    % Prevent overflowing lines due to hard-to-break entities
    \sloppy
    % Setup hyperref package
    \hypersetup{
      breaklinks=true,  % so long urls are correctly broken across lines
      colorlinks=true,
      urlcolor=urlcolor,
      linkcolor=linkcolor,
      citecolor=citecolor,
      }
    % Slightly bigger margins than the latex defaults
    
    \geometry{verbose,tmargin=1in,bmargin=1in,lmargin=1in,rmargin=1in}
    
    

\begin{document}
    
    \maketitle
    
    

    
    \section{Introduction and Setup}\label{introduction-and-setup}

\subsection{Binomial Option Pricing
Model}\label{binomial-option-pricing-model}

The Binomial Option Pricing Model is a discrete-time framework for
valuing options. It represents the price evolution of the underlying
asset as a tree of possible values over time.

Each node \((i, j)\) in the tree corresponds to:
\begin{enumerate}
\tightlist
    \item A specific time step \((i)\).
    \item A potential price outcome \((S_{i,j})\).
\end{enumerate}

At each node, the option value depends on:
\begin{enumerate}
\tightlist
    \item The \textbf{payoff} at expiration.
    \item The discounted expected value of future payoffs.
\end{enumerate}

\subsection{Why Choosing Parameters is
Important}\label{why-choosing-parameters-is-important}

Different variations of the binomial model affect:
\begin{enumerate}
\tightlist
    \item The up \((u)\) and down \((d)\) movement factors.
    \item The probability of an upward movement \((q)\).
    \item The convergence speed of the computed option price to the
theoretical value (e.g., Black-Scholes).
\end{enumerate}

This notebook explores four popular parameterization methods:
\begin{enumerate}
\tightlist
    \item \textbf{Cox, Ross, and Rubinstein (CRR):} Uses equal jump sizes for up
and down movements.
    \item \textbf{Jarrow and Rudd (JR):} Adjusts for a
risk-neutral measure while maintaining equal probabilities.
    \item \textbf{Equal Probabilities (EQP):} Modifies up and down jumps for a
logarithmic asset pricing tree.
    \item \textbf{Trigeorgis (TRG):} Ensures
equal jump sizes under a logarithmic asset pricing tree.
\end{enumerate}

\subsection{Goal}\label{goal}

We aim to:
\begin{enumerate}
\tightlist
    \item Implement these parameterization methods.
    \item Compare their
convergence characteristics against the Black-Scholes formula for
pricing European Call options.
\end{enumerate}

    \begin{tcolorbox}[breakable, size=fbox, boxrule=1pt, pad at break*=1mm,colback=cellbackground, colframe=cellborder]
\prompt{In}{incolor}{1}{\boxspacing}
\begin{Verbatim}[commandchars=\\\{\}]
\PY{k+kn}{import} \PY{n+nn}{numpy} \PY{k}{as} \PY{n+nn}{np}  \PY{c+c1}{\PYZsh{} For numerical operations}

\PY{c+c1}{\PYZsh{} Initialize parameters}
\PY{n}{S0} \PY{o}{=} \PY{l+m+mi}{100}       \PY{c+c1}{\PYZsh{} Initial stock price}
\PY{n}{K} \PY{o}{=} \PY{l+m+mi}{110}        \PY{c+c1}{\PYZsh{} Strike price}
\PY{n}{T} \PY{o}{=} \PY{l+m+mf}{0.5}        \PY{c+c1}{\PYZsh{} Time to maturity (in years)}
\PY{n}{r} \PY{o}{=} \PY{l+m+mf}{0.06}       \PY{c+c1}{\PYZsh{} Annual risk\PYZhy{}free rate}
\PY{n}{N} \PY{o}{=} \PY{l+m+mi}{100}        \PY{c+c1}{\PYZsh{} Number of time steps}
\PY{n}{sigma} \PY{o}{=} \PY{l+m+mf}{0.3}    \PY{c+c1}{\PYZsh{} Annualized stock price volatility}
\PY{n}{opttype} \PY{o}{=} \PY{l+s+s1}{\PYZsq{}}\PY{l+s+s1}{C}\PY{l+s+s1}{\PYZsq{}}  \PY{c+c1}{\PYZsh{} Option type: \PYZsq{}C\PYZsq{} for Call, \PYZsq{}P\PYZsq{} for Put}
\end{Verbatim}
\end{tcolorbox}

\pagebreak

    \section{Cox, Ross, and Rubinstein (CRR)
Method}\label{cox-ross-and-rubinstein-crr-method}

\subsection{Overview}\label{overview}

The CRR method is a classic parameterization for the binomial model. It
assumes:
\begin{enumerate}
\tightlist
    \item \textbf{Equal Jump Sizes:} The factors for up and down
movements are inverses: \[ u = e^{\sigma \sqrt{\Delta t}}, \quad d = \frac{1}{u} \]
    \item \textbf{Risk-Neutral Probabilities:} The
probability of an upward movement is: \[ q = \frac{e^{r \Delta t} - d}{u - d} \]
    \item \textbf{Option Pricing:} The
option price is calculated through backward induction.
\end{enumerate}

    \begin{tcolorbox}[breakable, size=fbox, boxrule=1pt, pad at break*=1mm,colback=cellbackground, colframe=cellborder]
\prompt{In}{incolor}{2}{\boxspacing}
\begin{Verbatim}[commandchars=\\\{\}]
\PY{k}{def} \PY{n+nf}{CRR\PYZus{}method}\PY{p}{(}\PY{n}{K}\PY{p}{,} \PY{n}{T}\PY{p}{,} \PY{n}{S0}\PY{p}{,} \PY{n}{r}\PY{p}{,} \PY{n}{N}\PY{p}{,} \PY{n}{sigma}\PY{p}{,} \PY{n}{opttype}\PY{o}{=}\PY{l+s+s1}{\PYZsq{}}\PY{l+s+s1}{C}\PY{l+s+s1}{\PYZsq{}}\PY{p}{)}\PY{p}{:}
    \PY{c+c1}{\PYZsh{} Precompute constants}
    \PY{n}{dt} \PY{o}{=} \PY{n}{T} \PY{o}{/} \PY{n}{N}  \PY{c+c1}{\PYZsh{} Time step}
    \PY{n}{u} \PY{o}{=} \PY{n}{np}\PY{o}{.}\PY{n}{exp}\PY{p}{(}\PY{n}{sigma} \PY{o}{*} \PY{n}{np}\PY{o}{.}\PY{n}{sqrt}\PY{p}{(}\PY{n}{dt}\PY{p}{)}\PY{p}{)}  \PY{c+c1}{\PYZsh{} Up factor}
    \PY{n}{d} \PY{o}{=} \PY{l+m+mi}{1} \PY{o}{/} \PY{n}{u}  \PY{c+c1}{\PYZsh{} Down factor}
    \PY{n}{q} \PY{o}{=} \PY{p}{(}\PY{n}{np}\PY{o}{.}\PY{n}{exp}\PY{p}{(}\PY{n}{r} \PY{o}{*} \PY{n}{dt}\PY{p}{)} \PY{o}{\PYZhy{}} \PY{n}{d}\PY{p}{)} \PY{o}{/} \PY{p}{(}\PY{n}{u} \PY{o}{\PYZhy{}} \PY{n}{d}\PY{p}{)}  \PY{c+c1}{\PYZsh{} Risk\PYZhy{}neutral probability}
    \PY{n}{disc} \PY{o}{=} \PY{n}{np}\PY{o}{.}\PY{n}{exp}\PY{p}{(}\PY{o}{\PYZhy{}}\PY{n}{r} \PY{o}{*} \PY{n}{dt}\PY{p}{)}  \PY{c+c1}{\PYZsh{} Discount factor}

    \PY{c+c1}{\PYZsh{} Initialize asset prices at maturity}
    \PY{n}{S} \PY{o}{=} \PY{n}{np}\PY{o}{.}\PY{n}{zeros}\PY{p}{(}\PY{n}{N} \PY{o}{+} \PY{l+m+mi}{1}\PY{p}{)}
    \PY{n}{S}\PY{p}{[}\PY{l+m+mi}{0}\PY{p}{]} \PY{o}{=} \PY{n}{S0} \PY{o}{*} \PY{n}{d}\PY{o}{*}\PY{o}{*}\PY{n}{N}  \PY{c+c1}{\PYZsh{} Lowest price at maturity}
    \PY{k}{for} \PY{n}{j} \PY{o+ow}{in} \PY{n+nb}{range}\PY{p}{(}\PY{l+m+mi}{1}\PY{p}{,} \PY{n}{N} \PY{o}{+} \PY{l+m+mi}{1}\PY{p}{)}\PY{p}{:}
        \PY{n}{S}\PY{p}{[}\PY{n}{j}\PY{p}{]} \PY{o}{=} \PY{n}{S}\PY{p}{[}\PY{n}{j} \PY{o}{\PYZhy{}} \PY{l+m+mi}{1}\PY{p}{]} \PY{o}{*} \PY{n}{u} \PY{o}{/} \PY{n}{d}

    \PY{c+c1}{\PYZsh{} Initialize option values at maturity}
    \PY{n}{C} \PY{o}{=} \PY{n}{np}\PY{o}{.}\PY{n}{zeros}\PY{p}{(}\PY{n}{N} \PY{o}{+} \PY{l+m+mi}{1}\PY{p}{)}
    \PY{k}{for} \PY{n}{j} \PY{o+ow}{in} \PY{n+nb}{range}\PY{p}{(}\PY{n}{N} \PY{o}{+} \PY{l+m+mi}{1}\PY{p}{)}\PY{p}{:}
        \PY{k}{if} \PY{n}{opttype} \PY{o}{==} \PY{l+s+s1}{\PYZsq{}}\PY{l+s+s1}{C}\PY{l+s+s1}{\PYZsq{}}\PY{p}{:}
            \PY{n}{C}\PY{p}{[}\PY{n}{j}\PY{p}{]} \PY{o}{=} \PY{n+nb}{max}\PY{p}{(}\PY{l+m+mi}{0}\PY{p}{,} \PY{n}{S}\PY{p}{[}\PY{n}{j}\PY{p}{]} \PY{o}{\PYZhy{}} \PY{n}{K}\PY{p}{)}  \PY{c+c1}{\PYZsh{} Call option payoff}
        \PY{k}{else}\PY{p}{:}
            \PY{n}{C}\PY{p}{[}\PY{n}{j}\PY{p}{]} \PY{o}{=} \PY{n+nb}{max}\PY{p}{(}\PY{l+m+mi}{0}\PY{p}{,} \PY{n}{K} \PY{o}{\PYZhy{}} \PY{n}{S}\PY{p}{[}\PY{n}{j}\PY{p}{]}\PY{p}{)}  \PY{c+c1}{\PYZsh{} Put option payoff}

    \PY{c+c1}{\PYZsh{} Backward induction}
    \PY{k}{for} \PY{n}{i} \PY{o+ow}{in} \PY{n+nb}{range}\PY{p}{(}\PY{n}{N} \PY{o}{\PYZhy{}} \PY{l+m+mi}{1}\PY{p}{,} \PY{o}{\PYZhy{}}\PY{l+m+mi}{1}\PY{p}{,} \PY{o}{\PYZhy{}}\PY{l+m+mi}{1}\PY{p}{)}\PY{p}{:}
        \PY{k}{for} \PY{n}{j} \PY{o+ow}{in} \PY{n+nb}{range}\PY{p}{(}\PY{n}{i} \PY{o}{+} \PY{l+m+mi}{1}\PY{p}{)}\PY{p}{:}
            \PY{n}{C}\PY{p}{[}\PY{n}{j}\PY{p}{]} \PY{o}{=} \PY{n}{disc} \PY{o}{*} \PY{p}{(}\PY{n}{q} \PY{o}{*} \PY{n}{C}\PY{p}{[}\PY{n}{j} \PY{o}{+} \PY{l+m+mi}{1}\PY{p}{]} \PY{o}{+} \PY{p}{(}\PY{l+m+mi}{1} \PY{o}{\PYZhy{}} \PY{n}{q}\PY{p}{)} \PY{o}{*} \PY{n}{C}\PY{p}{[}\PY{n}{j}\PY{p}{]}\PY{p}{)}
    
    \PY{k}{return} \PY{n}{C}\PY{p}{[}\PY{l+m+mi}{0}\PY{p}{]}

\PY{c+c1}{\PYZsh{} Example usage}
\PY{n}{CRR\PYZus{}method}\PY{p}{(}\PY{n}{K}\PY{p}{,} \PY{n}{T}\PY{p}{,} \PY{n}{S0}\PY{p}{,} \PY{n}{r}\PY{p}{,} \PY{n}{N}\PY{p}{,} \PY{n}{sigma}\PY{p}{,} \PY{n}{opttype}\PY{o}{=}\PY{l+s+s1}{\PYZsq{}}\PY{l+s+s1}{C}\PY{l+s+s1}{\PYZsq{}}\PY{p}{)}
\end{Verbatim}
\end{tcolorbox}

            \begin{tcolorbox}[breakable, size=fbox, boxrule=.5pt, pad at break*=1mm, opacityfill=0]
\prompt{Out}{outcolor}{2}{\boxspacing}
\begin{Verbatim}[commandchars=\\\{\}]
5.77342630682585
\end{Verbatim}
\end{tcolorbox}

\pagebreak
        
    \section{Jarrow and Rudd (JR)
Method}\label{jarrow-and-rudd-jr-method}

\subsection{Overview}\label{overview}

The JR method modifies the CRR approach by introducing a drift
adjustment \((\nu)\) for the risk-neutral measure: \( (\nu = r - \frac{\sigma^2}{2}) \) Under the JR method:
\begin{enumerate}
    \item \textbf{Up and Down Factors:} These incorporate the drift: \[
u = e^{\nu \Delta t + \sigma \sqrt{\Delta t}}, \quad d = e^{\nu \Delta t - \sigma \sqrt{\Delta t}}
\]
    \item \textbf{Equal Probabilities:} The upward
and downward probabilities are set to: \( q = 0.5 \)
\end{enumerate}

    \begin{tcolorbox}[breakable, size=fbox, boxrule=1pt, pad at break*=1mm,colback=cellbackground, colframe=cellborder]
\prompt{In}{incolor}{3}{\boxspacing}
\begin{Verbatim}[commandchars=\\\{\}]
\PY{k}{def} \PY{n+nf}{JR\PYZus{}method}\PY{p}{(}\PY{n}{K}\PY{p}{,} \PY{n}{T}\PY{p}{,} \PY{n}{S0}\PY{p}{,} \PY{n}{r}\PY{p}{,} \PY{n}{N}\PY{p}{,} \PY{n}{sigma}\PY{p}{,} \PY{n}{opttype}\PY{o}{=}\PY{l+s+s1}{\PYZsq{}}\PY{l+s+s1}{C}\PY{l+s+s1}{\PYZsq{}}\PY{p}{)}\PY{p}{:}
    \PY{c+c1}{\PYZsh{} Precompute constants}
    \PY{n}{dt} \PY{o}{=} \PY{n}{T} \PY{o}{/} \PY{n}{N}
    \PY{n}{nu} \PY{o}{=} \PY{n}{r} \PY{o}{\PYZhy{}} \PY{l+m+mf}{0.5} \PY{o}{*} \PY{n}{sigma}\PY{o}{*}\PY{o}{*}\PY{l+m+mi}{2}  \PY{c+c1}{\PYZsh{} Drift adjustment}
    \PY{n}{u} \PY{o}{=} \PY{n}{np}\PY{o}{.}\PY{n}{exp}\PY{p}{(}\PY{n}{nu} \PY{o}{*} \PY{n}{dt} \PY{o}{+} \PY{n}{sigma} \PY{o}{*} \PY{n}{np}\PY{o}{.}\PY{n}{sqrt}\PY{p}{(}\PY{n}{dt}\PY{p}{)}\PY{p}{)}  \PY{c+c1}{\PYZsh{} Up factor}
    \PY{n}{d} \PY{o}{=} \PY{n}{np}\PY{o}{.}\PY{n}{exp}\PY{p}{(}\PY{n}{nu} \PY{o}{*} \PY{n}{dt} \PY{o}{\PYZhy{}} \PY{n}{sigma} \PY{o}{*} \PY{n}{np}\PY{o}{.}\PY{n}{sqrt}\PY{p}{(}\PY{n}{dt}\PY{p}{)}\PY{p}{)}  \PY{c+c1}{\PYZsh{} Down factor}
    \PY{n}{q} \PY{o}{=} \PY{l+m+mf}{0.5}  \PY{c+c1}{\PYZsh{} Equal probabilities}
    \PY{n}{disc} \PY{o}{=} \PY{n}{np}\PY{o}{.}\PY{n}{exp}\PY{p}{(}\PY{o}{\PYZhy{}}\PY{n}{r} \PY{o}{*} \PY{n}{dt}\PY{p}{)}  \PY{c+c1}{\PYZsh{} Discount factor}

    \PY{c+c1}{\PYZsh{} Initialize asset prices at maturity}
    \PY{n}{S} \PY{o}{=} \PY{n}{np}\PY{o}{.}\PY{n}{zeros}\PY{p}{(}\PY{n}{N} \PY{o}{+} \PY{l+m+mi}{1}\PY{p}{)}
    \PY{n}{S}\PY{p}{[}\PY{l+m+mi}{0}\PY{p}{]} \PY{o}{=} \PY{n}{S0} \PY{o}{*} \PY{n}{d}\PY{o}{*}\PY{o}{*}\PY{n}{N}
    \PY{k}{for} \PY{n}{j} \PY{o+ow}{in} \PY{n+nb}{range}\PY{p}{(}\PY{l+m+mi}{1}\PY{p}{,} \PY{n}{N} \PY{o}{+} \PY{l+m+mi}{1}\PY{p}{)}\PY{p}{:}
        \PY{n}{S}\PY{p}{[}\PY{n}{j}\PY{p}{]} \PY{o}{=} \PY{n}{S}\PY{p}{[}\PY{n}{j} \PY{o}{\PYZhy{}} \PY{l+m+mi}{1}\PY{p}{]} \PY{o}{*} \PY{n}{u} \PY{o}{/} \PY{n}{d}

    \PY{c+c1}{\PYZsh{} Initialize option values at maturity}
    \PY{n}{C} \PY{o}{=} \PY{n}{np}\PY{o}{.}\PY{n}{zeros}\PY{p}{(}\PY{n}{N} \PY{o}{+} \PY{l+m+mi}{1}\PY{p}{)}
    \PY{k}{for} \PY{n}{j} \PY{o+ow}{in} \PY{n+nb}{range}\PY{p}{(}\PY{n}{N} \PY{o}{+} \PY{l+m+mi}{1}\PY{p}{)}\PY{p}{:}
        \PY{k}{if} \PY{n}{opttype} \PY{o}{==} \PY{l+s+s1}{\PYZsq{}}\PY{l+s+s1}{C}\PY{l+s+s1}{\PYZsq{}}\PY{p}{:}
            \PY{n}{C}\PY{p}{[}\PY{n}{j}\PY{p}{]} \PY{o}{=} \PY{n+nb}{max}\PY{p}{(}\PY{l+m+mi}{0}\PY{p}{,} \PY{n}{S}\PY{p}{[}\PY{n}{j}\PY{p}{]} \PY{o}{\PYZhy{}} \PY{n}{K}\PY{p}{)}
        \PY{k}{else}\PY{p}{:}
            \PY{n}{C}\PY{p}{[}\PY{n}{j}\PY{p}{]} \PY{o}{=} \PY{n+nb}{max}\PY{p}{(}\PY{l+m+mi}{0}\PY{p}{,} \PY{n}{K} \PY{o}{\PYZhy{}} \PY{n}{S}\PY{p}{[}\PY{n}{j}\PY{p}{]}\PY{p}{)}

    \PY{c+c1}{\PYZsh{} Backward induction}
    \PY{k}{for} \PY{n}{i} \PY{o+ow}{in} \PY{n+nb}{range}\PY{p}{(}\PY{n}{N} \PY{o}{\PYZhy{}} \PY{l+m+mi}{1}\PY{p}{,} \PY{o}{\PYZhy{}}\PY{l+m+mi}{1}\PY{p}{,} \PY{o}{\PYZhy{}}\PY{l+m+mi}{1}\PY{p}{)}\PY{p}{:}
        \PY{k}{for} \PY{n}{j} \PY{o+ow}{in} \PY{n+nb}{range}\PY{p}{(}\PY{n}{i} \PY{o}{+} \PY{l+m+mi}{1}\PY{p}{)}\PY{p}{:}
            \PY{n}{C}\PY{p}{[}\PY{n}{j}\PY{p}{]} \PY{o}{=} \PY{n}{disc} \PY{o}{*} \PY{p}{(}\PY{n}{q} \PY{o}{*} \PY{n}{C}\PY{p}{[}\PY{n}{j} \PY{o}{+} \PY{l+m+mi}{1}\PY{p}{]} \PY{o}{+} \PY{p}{(}\PY{l+m+mi}{1} \PY{o}{\PYZhy{}} \PY{n}{q}\PY{p}{)} \PY{o}{*} \PY{n}{C}\PY{p}{[}\PY{n}{j}\PY{p}{]}\PY{p}{)}
    
    \PY{k}{return} \PY{n}{C}\PY{p}{[}\PY{l+m+mi}{0}\PY{p}{]}

\PY{c+c1}{\PYZsh{} Example usage}
\PY{n}{JR\PYZus{}method}\PY{p}{(}\PY{n}{K}\PY{p}{,} \PY{n}{T}\PY{p}{,} \PY{n}{S0}\PY{p}{,} \PY{n}{r}\PY{p}{,} \PY{n}{N}\PY{p}{,} \PY{n}{sigma}\PY{p}{,} \PY{n}{opttype}\PY{o}{=}\PY{l+s+s1}{\PYZsq{}}\PY{l+s+s1}{C}\PY{l+s+s1}{\PYZsq{}}\PY{p}{)}
\end{Verbatim}
\end{tcolorbox}

            \begin{tcolorbox}[breakable, size=fbox, boxrule=.5pt, pad at break*=1mm, opacityfill=0]
\prompt{Out}{outcolor}{3}{\boxspacing}
\begin{Verbatim}[commandchars=\\\{\}]
5.754089414567556
\end{Verbatim}
\end{tcolorbox}

\pagebreak
        
    \section{Equal Probabilities (EQP)
Method}\label{equal-probabilities-eqp-method}

\subsection{Overview}\label{overview}

The EQP method focuses on equal risk-neutral probabilities \((q = 0.5)\)
while adjusting the up and down factors \( dx_u, dx_d \) for a
logarithmic asset pricing tree. It incorporates both drift and
volatility into these factors: \[
dx_u = 0.5 \nu \Delta t + 0.5 \sqrt{4 \sigma^2 \Delta t - 3 \nu^2 \Delta t^2}
\]
\[
dx_d = 1.5 \nu \Delta t - 0.5 \sqrt{4 \sigma^2 \Delta t - 3 \nu^2 \Delta t^2}
\]

Where:
\begin{itemize}
    \item \( \nu = r - \frac{\sigma^2}{2} \): Drift adjustment.
    \item \(dx_u\) and
\(dx_d\) define the multiplicative factors for up and down movements in
the logarithmic price space.
\end{itemize}

\subsection{Key Characteristics}\label{key-characteristics}

\begin{enumerate}
\item
  \textbf{Stock Prices:} Adjusted for logarithmic movements: \[
S_{i,j} = S_0 \exp\left(j \cdot dx_u + (i-j) \cdot dx_d\right)
\]

\item
  \textbf{Option Pricing:} Backward induction is performed with equal
  probabilities: \[
q = p_u = 0.5, \quad p_d = 1 - q
\]

\end{enumerate}

    \begin{tcolorbox}[breakable, size=fbox, boxrule=1pt, pad at break*=1mm,colback=cellbackground, colframe=cellborder]
\prompt{In}{incolor}{4}{\boxspacing}
\begin{Verbatim}[commandchars=\\\{\}]
\PY{k}{def} \PY{n+nf}{EQP\PYZus{}method}\PY{p}{(}\PY{n}{K}\PY{p}{,} \PY{n}{T}\PY{p}{,} \PY{n}{S0}\PY{p}{,} \PY{n}{r}\PY{p}{,} \PY{n}{N}\PY{p}{,} \PY{n}{sigma}\PY{p}{,} \PY{n}{opttype}\PY{o}{=}\PY{l+s+s1}{\PYZsq{}}\PY{l+s+s1}{C}\PY{l+s+s1}{\PYZsq{}}\PY{p}{)}\PY{p}{:}
    \PY{c+c1}{\PYZsh{} Precompute constants}
    \PY{n}{dt} \PY{o}{=} \PY{n}{T} \PY{o}{/} \PY{n}{N}
    \PY{n}{nu} \PY{o}{=} \PY{n}{r} \PY{o}{\PYZhy{}} \PY{l+m+mf}{0.5} \PY{o}{*} \PY{n}{sigma}\PY{o}{*}\PY{o}{*}\PY{l+m+mi}{2}
    \PY{n}{dxu} \PY{o}{=} \PY{l+m+mf}{0.5} \PY{o}{*} \PY{n}{nu} \PY{o}{*} \PY{n}{dt} \PY{o}{+} \PY{l+m+mf}{0.5} \PY{o}{*} \PY{n}{np}\PY{o}{.}\PY{n}{sqrt}\PY{p}{(}\PY{l+m+mi}{4} \PY{o}{*} \PY{n}{sigma}\PY{o}{*}\PY{o}{*}\PY{l+m+mi}{2} \PY{o}{*} \PY{n}{dt} \PY{o}{\PYZhy{}} \PY{l+m+mi}{3} \PY{o}{*} \PY{n}{nu}\PY{o}{*}\PY{o}{*}\PY{l+m+mi}{2} \PY{o}{*} \PY{n}{dt}\PY{o}{*}\PY{o}{*}\PY{l+m+mi}{2}\PY{p}{)}
    \PY{n}{dxd} \PY{o}{=} \PY{l+m+mf}{1.5} \PY{o}{*} \PY{n}{nu} \PY{o}{*} \PY{n}{dt} \PY{o}{\PYZhy{}} \PY{l+m+mf}{0.5} \PY{o}{*} \PY{n}{np}\PY{o}{.}\PY{n}{sqrt}\PY{p}{(}\PY{l+m+mi}{4} \PY{o}{*} \PY{n}{sigma}\PY{o}{*}\PY{o}{*}\PY{l+m+mi}{2} \PY{o}{*} \PY{n}{dt} \PY{o}{\PYZhy{}} \PY{l+m+mi}{3} \PY{o}{*} \PY{n}{nu}\PY{o}{*}\PY{o}{*}\PY{l+m+mi}{2} \PY{o}{*} \PY{n}{dt}\PY{o}{*}\PY{o}{*}\PY{l+m+mi}{2}\PY{p}{)}
    \PY{n}{pu} \PY{o}{=} \PY{l+m+mf}{0.5}  \PY{c+c1}{\PYZsh{} Equal probabilities}
    \PY{n}{pd} \PY{o}{=} \PY{l+m+mi}{1} \PY{o}{\PYZhy{}} \PY{n}{pu}
    \PY{n}{disc} \PY{o}{=} \PY{n}{np}\PY{o}{.}\PY{n}{exp}\PY{p}{(}\PY{o}{\PYZhy{}}\PY{n}{r} \PY{o}{*} \PY{n}{dt}\PY{p}{)}

    \PY{c+c1}{\PYZsh{} Initialize asset prices at maturity}
    \PY{n}{S} \PY{o}{=} \PY{n}{np}\PY{o}{.}\PY{n}{zeros}\PY{p}{(}\PY{n}{N} \PY{o}{+} \PY{l+m+mi}{1}\PY{p}{)}
    \PY{n}{S}\PY{p}{[}\PY{l+m+mi}{0}\PY{p}{]} \PY{o}{=} \PY{n}{S0} \PY{o}{*} \PY{n}{np}\PY{o}{.}\PY{n}{exp}\PY{p}{(}\PY{n}{N} \PY{o}{*} \PY{n}{dxd}\PY{p}{)}
    \PY{k}{for} \PY{n}{j} \PY{o+ow}{in} \PY{n+nb}{range}\PY{p}{(}\PY{l+m+mi}{1}\PY{p}{,} \PY{n}{N} \PY{o}{+} \PY{l+m+mi}{1}\PY{p}{)}\PY{p}{:}
        \PY{n}{S}\PY{p}{[}\PY{n}{j}\PY{p}{]} \PY{o}{=} \PY{n}{S}\PY{p}{[}\PY{n}{j} \PY{o}{\PYZhy{}} \PY{l+m+mi}{1}\PY{p}{]} \PY{o}{*} \PY{n}{np}\PY{o}{.}\PY{n}{exp}\PY{p}{(}\PY{n}{dxu} \PY{o}{\PYZhy{}} \PY{n}{dxd}\PY{p}{)}

    \PY{c+c1}{\PYZsh{} Initialize option values at maturity}
    \PY{n}{C} \PY{o}{=} \PY{n}{np}\PY{o}{.}\PY{n}{zeros}\PY{p}{(}\PY{n}{N} \PY{o}{+} \PY{l+m+mi}{1}\PY{p}{)}
    \PY{k}{for} \PY{n}{j} \PY{o+ow}{in} \PY{n+nb}{range}\PY{p}{(}\PY{n}{N} \PY{o}{+} \PY{l+m+mi}{1}\PY{p}{)}\PY{p}{:}
        \PY{k}{if} \PY{n}{opttype} \PY{o}{==} \PY{l+s+s1}{\PYZsq{}}\PY{l+s+s1}{C}\PY{l+s+s1}{\PYZsq{}}\PY{p}{:}
            \PY{n}{C}\PY{p}{[}\PY{n}{j}\PY{p}{]} \PY{o}{=} \PY{n+nb}{max}\PY{p}{(}\PY{l+m+mi}{0}\PY{p}{,} \PY{n}{S}\PY{p}{[}\PY{n}{j}\PY{p}{]} \PY{o}{\PYZhy{}} \PY{n}{K}\PY{p}{)}
        \PY{k}{else}\PY{p}{:}
            \PY{n}{C}\PY{p}{[}\PY{n}{j}\PY{p}{]} \PY{o}{=} \PY{n+nb}{max}\PY{p}{(}\PY{l+m+mi}{0}\PY{p}{,} \PY{n}{K} \PY{o}{\PYZhy{}} \PY{n}{S}\PY{p}{[}\PY{n}{j}\PY{p}{]}\PY{p}{)}

    \PY{c+c1}{\PYZsh{} Backward induction}
    \PY{k}{for} \PY{n}{i} \PY{o+ow}{in} \PY{n+nb}{range}\PY{p}{(}\PY{n}{N} \PY{o}{\PYZhy{}} \PY{l+m+mi}{1}\PY{p}{,} \PY{o}{\PYZhy{}}\PY{l+m+mi}{1}\PY{p}{,} \PY{o}{\PYZhy{}}\PY{l+m+mi}{1}\PY{p}{)}\PY{p}{:}
        \PY{k}{for} \PY{n}{j} \PY{o+ow}{in} \PY{n+nb}{range}\PY{p}{(}\PY{n}{i} \PY{o}{+} \PY{l+m+mi}{1}\PY{p}{)}\PY{p}{:}
            \PY{n}{C}\PY{p}{[}\PY{n}{j}\PY{p}{]} \PY{o}{=} \PY{n}{disc} \PY{o}{*} \PY{p}{(}\PY{n}{pu} \PY{o}{*} \PY{n}{C}\PY{p}{[}\PY{n}{j} \PY{o}{+} \PY{l+m+mi}{1}\PY{p}{]} \PY{o}{+} \PY{n}{pd} \PY{o}{*} \PY{n}{C}\PY{p}{[}\PY{n}{j}\PY{p}{]}\PY{p}{)}
    
    \PY{k}{return} \PY{n}{C}\PY{p}{[}\PY{l+m+mi}{0}\PY{p}{]}

\PY{c+c1}{\PYZsh{} Example usage}
\PY{n}{EQP\PYZus{}method}\PY{p}{(}\PY{n}{K}\PY{p}{,} \PY{n}{T}\PY{p}{,} \PY{n}{S0}\PY{p}{,} \PY{n}{r}\PY{p}{,} \PY{n}{N}\PY{p}{,} \PY{n}{sigma}\PY{p}{,} \PY{n}{opttype}\PY{o}{=}\PY{l+s+s1}{\PYZsq{}}\PY{l+s+s1}{C}\PY{l+s+s1}{\PYZsq{}}\PY{p}{)}
\end{Verbatim}
\end{tcolorbox}

            \begin{tcolorbox}[breakable, size=fbox, boxrule=.5pt, pad at break*=1mm, opacityfill=0]
\prompt{Out}{outcolor}{4}{\boxspacing}
\begin{Verbatim}[commandchars=\\\{\}]
5.7365844788666545
\end{Verbatim}
\end{tcolorbox}

\pagebreak
        
    \section{Trigeorgis (TRG) Method}\label{trigeorgis-trg-method}

\subsection{Overview}\label{overview}

The TRG method ensures \textbf{equal jump sizes} under a logarithmic
asset pricing tree while accounting for drift \((\nu)\) and volatility
\((\sigma)\): \[
dx_u = \sqrt{\sigma^2 \Delta t + \nu^2 \Delta t^2}, \quad dx_d = -dx_u
\]

The probabilities for upward and downward movements are: \[
p_u = 0.5 + 0.5 \frac{\nu \Delta t}{dx_u}, \quad p_d = 1 - p_u
\]

\subsection{Key Characteristics}\label{key-characteristics}

\begin{enumerate}
\item
  \textbf{Stock Prices:} Adjusted for logarithmic jumps: \[
S_{i,j} = S_0 \exp\left(j \cdot dx_u + (i-j) \cdot dx_d\right)
\]
\item
  \textbf{Option Pricing:} Similar to other methods, backward induction
  is applied using the computed probabilities \((p_u)\) and \((p_d)\).
\end{enumerate}

    \begin{tcolorbox}[breakable, size=fbox, boxrule=1pt, pad at break*=1mm,colback=cellbackground, colframe=cellborder]
\prompt{In}{incolor}{5}{\boxspacing}
\begin{Verbatim}[commandchars=\\\{\}]
\PY{k}{def} \PY{n+nf}{TRG\PYZus{}method}\PY{p}{(}\PY{n}{K}\PY{p}{,} \PY{n}{T}\PY{p}{,} \PY{n}{S0}\PY{p}{,} \PY{n}{r}\PY{p}{,} \PY{n}{N}\PY{p}{,} \PY{n}{sigma}\PY{p}{,} \PY{n}{opttype}\PY{o}{=}\PY{l+s+s1}{\PYZsq{}}\PY{l+s+s1}{C}\PY{l+s+s1}{\PYZsq{}}\PY{p}{)}\PY{p}{:}
    \PY{c+c1}{\PYZsh{} Precompute constants}
    \PY{n}{dt} \PY{o}{=} \PY{n}{T} \PY{o}{/} \PY{n}{N}
    \PY{n}{nu} \PY{o}{=} \PY{n}{r} \PY{o}{\PYZhy{}} \PY{l+m+mf}{0.5} \PY{o}{*} \PY{n}{sigma}\PY{o}{*}\PY{o}{*}\PY{l+m+mi}{2}
    \PY{n}{dxu} \PY{o}{=} \PY{n}{np}\PY{o}{.}\PY{n}{sqrt}\PY{p}{(}\PY{n}{sigma}\PY{o}{*}\PY{o}{*}\PY{l+m+mi}{2} \PY{o}{*} \PY{n}{dt} \PY{o}{+} \PY{n}{nu}\PY{o}{*}\PY{o}{*}\PY{l+m+mi}{2} \PY{o}{*} \PY{n}{dt}\PY{o}{*}\PY{o}{*}\PY{l+m+mi}{2}\PY{p}{)}
    \PY{n}{dxd} \PY{o}{=} \PY{o}{\PYZhy{}}\PY{n}{dxu}
    \PY{n}{pu} \PY{o}{=} \PY{l+m+mf}{0.5} \PY{o}{+} \PY{l+m+mf}{0.5} \PY{o}{*} \PY{n}{nu} \PY{o}{*} \PY{n}{dt} \PY{o}{/} \PY{n}{dxu}
    \PY{n}{pd} \PY{o}{=} \PY{l+m+mi}{1} \PY{o}{\PYZhy{}} \PY{n}{pu}
    \PY{n}{disc} \PY{o}{=} \PY{n}{np}\PY{o}{.}\PY{n}{exp}\PY{p}{(}\PY{o}{\PYZhy{}}\PY{n}{r} \PY{o}{*} \PY{n}{dt}\PY{p}{)}

    \PY{c+c1}{\PYZsh{} Initialize asset prices at maturity}
    \PY{n}{S} \PY{o}{=} \PY{n}{np}\PY{o}{.}\PY{n}{zeros}\PY{p}{(}\PY{n}{N} \PY{o}{+} \PY{l+m+mi}{1}\PY{p}{)}
    \PY{n}{S}\PY{p}{[}\PY{l+m+mi}{0}\PY{p}{]} \PY{o}{=} \PY{n}{S0} \PY{o}{*} \PY{n}{np}\PY{o}{.}\PY{n}{exp}\PY{p}{(}\PY{n}{N} \PY{o}{*} \PY{n}{dxd}\PY{p}{)}
    \PY{k}{for} \PY{n}{j} \PY{o+ow}{in} \PY{n+nb}{range}\PY{p}{(}\PY{l+m+mi}{1}\PY{p}{,} \PY{n}{N} \PY{o}{+} \PY{l+m+mi}{1}\PY{p}{)}\PY{p}{:}
        \PY{n}{S}\PY{p}{[}\PY{n}{j}\PY{p}{]} \PY{o}{=} \PY{n}{S}\PY{p}{[}\PY{n}{j} \PY{o}{\PYZhy{}} \PY{l+m+mi}{1}\PY{p}{]} \PY{o}{*} \PY{n}{np}\PY{o}{.}\PY{n}{exp}\PY{p}{(}\PY{n}{dxu} \PY{o}{\PYZhy{}} \PY{n}{dxd}\PY{p}{)}

    \PY{c+c1}{\PYZsh{} Initialize option values at maturity}
    \PY{n}{C} \PY{o}{=} \PY{n}{np}\PY{o}{.}\PY{n}{zeros}\PY{p}{(}\PY{n}{N} \PY{o}{+} \PY{l+m+mi}{1}\PY{p}{)}
    \PY{k}{for} \PY{n}{j} \PY{o+ow}{in} \PY{n+nb}{range}\PY{p}{(}\PY{n}{N} \PY{o}{+} \PY{l+m+mi}{1}\PY{p}{)}\PY{p}{:}
        \PY{k}{if} \PY{n}{opttype} \PY{o}{==} \PY{l+s+s1}{\PYZsq{}}\PY{l+s+s1}{C}\PY{l+s+s1}{\PYZsq{}}\PY{p}{:}
            \PY{n}{C}\PY{p}{[}\PY{n}{j}\PY{p}{]} \PY{o}{=} \PY{n+nb}{max}\PY{p}{(}\PY{l+m+mi}{0}\PY{p}{,} \PY{n}{S}\PY{p}{[}\PY{n}{j}\PY{p}{]} \PY{o}{\PYZhy{}} \PY{n}{K}\PY{p}{)}
        \PY{k}{else}\PY{p}{:}
            \PY{n}{C}\PY{p}{[}\PY{n}{j}\PY{p}{]} \PY{o}{=} \PY{n+nb}{max}\PY{p}{(}\PY{l+m+mi}{0}\PY{p}{,} \PY{n}{K} \PY{o}{\PYZhy{}} \PY{n}{S}\PY{p}{[}\PY{n}{j}\PY{p}{]}\PY{p}{)}

    \PY{c+c1}{\PYZsh{} Backward induction}
    \PY{k}{for} \PY{n}{i} \PY{o+ow}{in} \PY{n+nb}{range}\PY{p}{(}\PY{n}{N} \PY{o}{\PYZhy{}} \PY{l+m+mi}{1}\PY{p}{,} \PY{o}{\PYZhy{}}\PY{l+m+mi}{1}\PY{p}{,} \PY{o}{\PYZhy{}}\PY{l+m+mi}{1}\PY{p}{)}\PY{p}{:}
        \PY{k}{for} \PY{n}{j} \PY{o+ow}{in} \PY{n+nb}{range}\PY{p}{(}\PY{n}{i} \PY{o}{+} \PY{l+m+mi}{1}\PY{p}{)}\PY{p}{:}
            \PY{n}{C}\PY{p}{[}\PY{n}{j}\PY{p}{]} \PY{o}{=} \PY{n}{disc} \PY{o}{*} \PY{p}{(}\PY{n}{pu} \PY{o}{*} \PY{n}{C}\PY{p}{[}\PY{n}{j} \PY{o}{+} \PY{l+m+mi}{1}\PY{p}{]} \PY{o}{+} \PY{n}{pd} \PY{o}{*} \PY{n}{C}\PY{p}{[}\PY{n}{j}\PY{p}{]}\PY{p}{)}
    
    \PY{k}{return} \PY{n}{C}\PY{p}{[}\PY{l+m+mi}{0}\PY{p}{]}

\PY{c+c1}{\PYZsh{} Example usage}
\PY{n}{TRG\PYZus{}method}\PY{p}{(}\PY{n}{K}\PY{p}{,} \PY{n}{T}\PY{p}{,} \PY{n}{S0}\PY{p}{,} \PY{n}{r}\PY{p}{,} \PY{n}{N}\PY{p}{,} \PY{n}{sigma}\PY{p}{,} \PY{n}{opttype}\PY{o}{=}\PY{l+s+s1}{\PYZsq{}}\PY{l+s+s1}{C}\PY{l+s+s1}{\PYZsq{}}\PY{p}{)}
\end{Verbatim}
\end{tcolorbox}

            \begin{tcolorbox}[breakable, size=fbox, boxrule=.5pt, pad at break*=1mm, opacityfill=0]
\prompt{Out}{outcolor}{5}{\boxspacing}
\begin{Verbatim}[commandchars=\\\{\}]
5.773359020180677
\end{Verbatim}
\end{tcolorbox}

\pagebreak
        
    \section{Comparison of Methods}\label{comparison-of-methods}

\subsection{Objective}\label{objective}

Evaluate the convergence of each parameterization method as the number
of time steps \((N)\) increases. The benchmark is the
\textbf{Black-Scholes formula} for pricing European Call options.

\subsection{Methodology}\label{methodology}

\begin{enumerate}
\item
  Compute the option price for each method \(CRR, JR, EQP, TRG\)
  over a range of \((N)\).
\item
  Compare against the theoretical Black-Scholes price.
\end{enumerate}

\subsection{Visualization}\label{visualization}

\begin{itemize}
\item
  Plot the computed prices for each method.
\item
  Highlight how quickly each method converges to the Black-Scholes
  price.
\end{itemize}

    \begin{tcolorbox}[breakable, size=fbox, boxrule=1pt, pad at break*=1mm,colback=cellbackground, colframe=cellborder]
\prompt{In}{incolor}{13}{\boxspacing}
\begin{Verbatim}[commandchars=\\\{\}]
\PY{k+kn}{from} \PY{n+nn}{py\PYZus{}vollib}\PY{n+nn}{.}\PY{n+nn}{black\PYZus{}scholes} \PY{k+kn}{import} \PY{n}{black\PYZus{}scholes} \PY{k}{as} \PY{n}{bs}
\PY{k+kn}{import} \PY{n+nn}{matplotlib}\PY{n+nn}{.}\PY{n+nn}{pyplot} \PY{k}{as} \PY{n+nn}{plt}

\PY{n}{plt}\PY{o}{.}\PY{n}{style}\PY{o}{.}\PY{n}{use}\PY{p}{(}\PY{l+s+s1}{\PYZsq{}}\PY{l+s+s1}{ggplot}\PY{l+s+s1}{\PYZsq{}}\PY{p}{)}

\PY{c+c1}{\PYZsh{} Compute option prices for different time steps}
\PY{n}{CRR}\PY{p}{,} \PY{n}{JR}\PY{p}{,} \PY{n}{EQP}\PY{p}{,} \PY{n}{TRG} \PY{o}{=} \PY{p}{[}\PY{p}{]}\PY{p}{,} \PY{p}{[}\PY{p}{]}\PY{p}{,} \PY{p}{[}\PY{p}{]}\PY{p}{,} \PY{p}{[}\PY{p}{]}
\PY{n}{periods} \PY{o}{=} \PY{n+nb}{range}\PY{p}{(}\PY{l+m+mi}{10}\PY{p}{,} \PY{l+m+mi}{500}\PY{p}{,} \PY{l+m+mi}{10}\PY{p}{)}

\PY{k}{for} \PY{n}{N} \PY{o+ow}{in} \PY{n}{periods}\PY{p}{:}
    \PY{n}{CRR}\PY{o}{.}\PY{n}{append}\PY{p}{(}\PY{n}{CRR\PYZus{}method}\PY{p}{(}\PY{n}{K}\PY{p}{,} \PY{n}{T}\PY{p}{,} \PY{n}{S0}\PY{p}{,} \PY{n}{r}\PY{p}{,} \PY{n}{N}\PY{p}{,} \PY{n}{sigma}\PY{p}{,} \PY{n}{opttype}\PY{o}{=}\PY{l+s+s1}{\PYZsq{}}\PY{l+s+s1}{C}\PY{l+s+s1}{\PYZsq{}}\PY{p}{)}\PY{p}{)}
    \PY{n}{JR}\PY{o}{.}\PY{n}{append}\PY{p}{(}\PY{n}{JR\PYZus{}method}\PY{p}{(}\PY{n}{K}\PY{p}{,} \PY{n}{T}\PY{p}{,} \PY{n}{S0}\PY{p}{,} \PY{n}{r}\PY{p}{,} \PY{n}{N}\PY{p}{,} \PY{n}{sigma}\PY{p}{,} \PY{n}{opttype}\PY{o}{=}\PY{l+s+s1}{\PYZsq{}}\PY{l+s+s1}{C}\PY{l+s+s1}{\PYZsq{}}\PY{p}{)}\PY{p}{)}
    \PY{n}{EQP}\PY{o}{.}\PY{n}{append}\PY{p}{(}\PY{n}{EQP\PYZus{}method}\PY{p}{(}\PY{n}{K}\PY{p}{,} \PY{n}{T}\PY{p}{,} \PY{n}{S0}\PY{p}{,} \PY{n}{r}\PY{p}{,} \PY{n}{N}\PY{p}{,} \PY{n}{sigma}\PY{p}{,} \PY{n}{opttype}\PY{o}{=}\PY{l+s+s1}{\PYZsq{}}\PY{l+s+s1}{C}\PY{l+s+s1}{\PYZsq{}}\PY{p}{)}\PY{p}{)}
    \PY{n}{TRG}\PY{o}{.}\PY{n}{append}\PY{p}{(}\PY{n}{TRG\PYZus{}method}\PY{p}{(}\PY{n}{K}\PY{p}{,} \PY{n}{T}\PY{p}{,} \PY{n}{S0}\PY{p}{,} \PY{n}{r}\PY{p}{,} \PY{n}{N}\PY{p}{,} \PY{n}{sigma}\PY{p}{,} \PY{n}{opttype}\PY{o}{=}\PY{l+s+s1}{\PYZsq{}}\PY{l+s+s1}{C}\PY{l+s+s1}{\PYZsq{}}\PY{p}{)}\PY{p}{)}

\PY{c+c1}{\PYZsh{} Black\PYZhy{}Scholes price}
\PY{n}{BS} \PY{o}{=} \PY{p}{[}\PY{n}{bs}\PY{p}{(}\PY{l+s+s1}{\PYZsq{}}\PY{l+s+s1}{c}\PY{l+s+s1}{\PYZsq{}}\PY{p}{,} \PY{n}{S0}\PY{p}{,} \PY{n}{K}\PY{p}{,} \PY{n}{T}\PY{p}{,} \PY{n}{r}\PY{p}{,} \PY{n}{sigma}\PY{p}{)} \PY{k}{for} \PY{n}{\PYZus{}} \PY{o+ow}{in} \PY{n}{periods}\PY{p}{]}

\PY{c+c1}{\PYZsh{} Plot results}
\PY{n}{plt}\PY{o}{.}\PY{n}{plot}\PY{p}{(}\PY{n}{periods}\PY{p}{,} \PY{n}{CRR}\PY{p}{,} \PY{n}{label}\PY{o}{=}\PY{l+s+s1}{\PYZsq{}}\PY{l+s+s1}{Cox\PYZhy{}Ross\PYZhy{}Rubinstein}\PY{l+s+s1}{\PYZsq{}}\PY{p}{)}
\PY{n}{plt}\PY{o}{.}\PY{n}{plot}\PY{p}{(}\PY{n}{periods}\PY{p}{,} \PY{n}{JR}\PY{p}{,} \PY{n}{label}\PY{o}{=}\PY{l+s+s1}{\PYZsq{}}\PY{l+s+s1}{Jarrow\PYZhy{}Rudd}\PY{l+s+s1}{\PYZsq{}}\PY{p}{)}
\PY{n}{plt}\PY{o}{.}\PY{n}{plot}\PY{p}{(}\PY{n}{periods}\PY{p}{,} \PY{n}{EQP}\PY{p}{,} \PY{n}{label}\PY{o}{=}\PY{l+s+s1}{\PYZsq{}}\PY{l+s+s1}{EQP}\PY{l+s+s1}{\PYZsq{}}\PY{p}{)}
\PY{n}{plt}\PY{o}{.}\PY{n}{plot}\PY{p}{(}\PY{n}{periods}\PY{p}{,} \PY{n}{TRG}\PY{p}{,} \PY{n}{label}\PY{o}{=}\PY{l+s+s1}{\PYZsq{}}\PY{l+s+s1}{Trigeorgis}\PY{l+s+s1}{\PYZsq{}}\PY{p}{,} \PY{n}{linestyle}\PY{o}{=}\PY{l+s+s1}{\PYZsq{}}\PY{l+s+s1}{\PYZhy{}\PYZhy{}}\PY{l+s+s1}{\PYZsq{}}\PY{p}{,} \PY{n}{color}\PY{o}{=}\PY{l+s+s1}{\PYZsq{}}\PY{l+s+s1}{r}\PY{l+s+s1}{\PYZsq{}}\PY{p}{)}
\PY{n}{plt}\PY{o}{.}\PY{n}{plot}\PY{p}{(}\PY{n}{periods}\PY{p}{,} \PY{n}{BS}\PY{p}{,} \PY{n}{label}\PY{o}{=}\PY{l+s+s1}{\PYZsq{}}\PY{l+s+s1}{Black\PYZhy{}Scholes}\PY{l+s+s1}{\PYZsq{}}\PY{p}{,} \PY{n}{linestyle}\PY{o}{=}\PY{l+s+s1}{\PYZsq{}}\PY{l+s+s1}{\PYZhy{}}\PY{l+s+s1}{\PYZsq{}}\PY{p}{,} \PY{n}{color}\PY{o}{=}\PY{l+s+s1}{\PYZsq{}}\PY{l+s+s1}{k}\PY{l+s+s1}{\PYZsq{}}\PY{p}{)}
\PY{n}{plt}\PY{o}{.}\PY{n}{legend}\PY{p}{(}\PY{n}{loc}\PY{o}{=}\PY{l+s+s1}{\PYZsq{}}\PY{l+s+s1}{upper right}\PY{l+s+s1}{\PYZsq{}}\PY{p}{)}
\PY{n}{plt}\PY{o}{.}\PY{n}{xlabel}\PY{p}{(}\PY{l+s+s1}{\PYZsq{}}\PY{l+s+s1}{Number of Time Steps (N)}\PY{l+s+s1}{\PYZsq{}}\PY{p}{)}
\PY{n}{plt}\PY{o}{.}\PY{n}{ylabel}\PY{p}{(}\PY{l+s+s1}{\PYZsq{}}\PY{l+s+s1}{Option Price}\PY{l+s+s1}{\PYZsq{}}\PY{p}{)}
\PY{n}{plt}\PY{o}{.}\PY{n}{title}\PY{p}{(}\PY{l+s+s1}{\PYZsq{}}\PY{l+s+s1}{Convergence of Binomial Methods to Black\PYZhy{}Scholes}\PY{l+s+s1}{\PYZsq{}}\PY{p}{)}
\PY{n}{plt}\PY{o}{.}\PY{n}{show}\PY{p}{(}\PY{p}{)}
\end{Verbatim}
\end{tcolorbox}

    \begin{center}
    \adjustimage{max size={0.9\linewidth}{0.9\paperheight}}{output_11_0.png}
    \end{center}
    { \hspace*{\fill} \\}
    

    % Add a bibliography block to the postdoc
    
    
    
\end{document}